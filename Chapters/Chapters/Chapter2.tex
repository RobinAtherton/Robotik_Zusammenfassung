\chapter{Software-Architekturen für mobile Robotersysteme}
\section{Probleme und Anforderungen}
\subsection{Definition mobile Roboter}
Unter einem Roboter verstehen wir eine frei programmierbare Maschine, die auf Basis von Umgebungssensordaten in geschlossener Regelung in Umgebungen agiert, die zur Zeit der Programmierung nicht genau bekannt und/oder dynamisch und oder nicht vollständig erfassbar sind.
\subsection{Umgebung mobiler Roboter}
Bei \textbf{mobilen Robtoern} ist die Umgebung im Detail \textbf{nicht bekannt und generell nicht kontrollierbar}
\begin{itemize}
	\item Alle Aktionen sind von der aktuellen Umgebung abhängig
	\item Details sind erst zum Zeitpunkt der Ausführung der Aktionen bekannt
	\item Mobile Roboter müssen in einer geschlossenen Regelung
		\subitem die Umgebung mit Sensoren erfassen
		\subitem die Daten auswerten
		\subitem Aktionen daraus planen
		\subitem Aktionen mittels Koordination der Aktuatoren umsetzen
\end{itemize}

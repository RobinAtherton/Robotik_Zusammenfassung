\chapter{Software-Architekturen für mobile Robotersysteme}
\section{Probleme und Anforderungen}
\subsection{Definition mobile Roboter}
\enquote{Unter einem Roboter verstehen wir eine frei programmierbare Maschine, die auf Basis von Umgebungssensordaten in geschlossener Regelung in Umgebungen agiert, die zur Zeit der Programmierung nicht genau bekannt und/oder dynamisch und oder nicht vollständig erfassbar sind.}
$\Rightarrow$ \textbf{Joachim Herzberg, Mobile Roboter}
\subsection{Umgebung mobiler Roboter}
Bei \textbf{mobilen Robtoern} ist die Umgebung im Detail \textbf{nicht bekannt und generell nicht kontrollierbar}
\begin{itemize}
	\item Alle Aktionen sind von der aktuellen Umgebung abhängig
	\item Details sind erst zum Zeitpunkt der Ausführung der Aktionen bekannt
	\item Mobile Roboter müssen in einer geschlossenen Regelung
		\subitem die Umgebung mit Sensoren erfassen
		\subitem die Daten auswerten
		\subitem Aktionen daraus planen
		\subitem Aktionen mittels Koordination der Aktuatoren umsetzen
\end{itemize}
\subsection{Roboterkontroll-Architekturen}
\paragraph{Herausforderungen}
\begin{itemize}
	\item Robotersystem besteht aus den Gebieten \textbf{Wahrnehmung}, \textbf{Planung} und \textbf{Handlung}
	\item Herausforderungen an eine Roboterkontroll-Architektur, sie muss:
	\subitem Sensorwerte erfassen und auswerten
	\subitem Pfade planen
	\subitem Hindernisse vermeiden
	\subitem Komplexe Algorithmen in langen Zeitzyklen ausführen
\end{itemize}
\paragraph{Probleme bei der Software-Erstellung zur Roboterkontrolle}
\begin{itemize}
	\item Roboter sind eingebettete Systeme, die in geschlossener Regelung laufen und die Sensorströme in \textbf{Echtzeit verarbeiten} müssen
	\item Untschiedliche Aufgaben --> Unterschiedliche Zeitzyklen
	\item Unterschiedlicher Zeitskalen --> kein standardisierter Kontroll- oder Datenfluss den die Architektur abbilden könnten
	\item Für etliche algorithmische Teilprobleme sind \textbf{keine effizienten Verfahren} bekannt
	\item \textbf{Prozessorkapazität ist begrenzt}
\end{itemize}
\subsection{Anforderungen an das Kontrollsystem eines autonomen Roboters}
\paragraph{Robustheit}
\begin{itemize}
	\item Die Umgebung des Systems kann sich ständig ändern
	\item Auf eine Umgebungsänderung sollte der Roboter sinnvoll reagieren und nicht verwirrt stehen bleiben.
	\item Verwendete Modelle der Umgebung sind ungenau.
\end{itemize}
\paragraph{Unterschiedliche Ziele}
\begin{itemize}
	\item Der Roboter verfolgt zu einem Zeitpunkt eventuell Ziele, die im Konflikt zueinander stehen.
	\item \textbf{Beispiel}: der Roboter soll ein bestimmtes Ziel ansteuern, dabei aber Hindernissen ausweichen.
\end{itemize}
\paragraph{Sensorwerte von mehreren Sensoren}
\begin{itemize}
	\item Sensordaten könen verrauscht sein
	\item Sensoren können fehlerhafte oder inkonsistente Messwerte liefern, weil der Sensor z.B. außerhalb seines Bereichs misst für den er zuständig ist und dies nicht überprüfen kann.
\end{itemize}
\paragraph{Erweiterbarkeit}
\begin{itemize}
	\item Wenn der Roboter neue Sensoren erhält, sollte dies leicht in das Programm integriert werden können.
\end{itemize}
\section{Mögliche Modelle}
\subsection{Klassisches Modell - der funktionale Ansatz}
Das \textbf{klassische Model} wird auch als hierarchisches Model oder funktionales Model bezeichnet.
Ist ein Top-Down Ansatz, besteht aus drei Abstraktionsebenen
\begin{itemize}
	\item Die unterste Ebene: \textbf{Pilot}
	\item Mittlere Eben: \textbf{Navigator}
	\item Oberste Ebene: \textbf{Planer}
\end{itemize}
\textbf{Sense-Think-Act-Cycle} oder \textbf{SMPA} (Sense - Model - Plan - Act).
\begin{itemize}
	\item Sensordaten, die vom Fahrzeut geliefert werden, werden in den zwei unteren Ebenen vorverarbeitet.
	\item Konstruktion oder Aktualisierung eines Weltmodells
	\item \textbf{Planer} ist die Basis aller Entscheidungen basieren auf dem zugrundelgenden Weltmodell
	\item Tatsächliche Fahrbefehle werden durch unterste Ebene ausgeführt
\end{itemize}
Zyklus wird ständig wiederholt $\Rightarrow$ wenn alle Ebenen richtig funktionieren resultiert daraus ein intelligentes Verhalten und die Erfüllung der Aufgabe.
%Kapitel 2 seite 3


\chapter{Geschichte}
\section{Industrieroboter}
Nach Definition der VDI-Richtlinie 2860 sind Industrieroboter universell einsetzbare Bewegungsautomaten mit meherern Achsen, deren Bewegungen hinsichtlich Bewegungsfolge und Wegen bzw. Winkel frei programmierbar und sensorgeführt sind.
\begin{itemize}
	\item Zeichnen sich durch \textbf{Schnelligkeit}, \textbf{Genauigkeit}, \textbf{Robustheit }und eine hohe \textbf{Traglast }aus.
	\item Einsatzgebiete: Schweißen, Kleben, Schneide, Lackieren
\end{itemize}
Zunehmend \textbf{kollaborative} Roboter, Cobots:
\begin{itemize}
	\item Industrieroboter, die mit Menschen gemeinsam arbeiten
	\item Nicht mehr durch Schutzeinrichtungen im Produktionsprozess von Menschen getrennt
	\item Nimmt Menschen wahr, verursacht keine Verletzungen
\end{itemize}

\section{Serviceroboter}
\subsection{Definition}
\begin{itemize}
	\item Ein \textbf{Serviceroboter} ist eine \textbf{frei programmierbare Bewegungseinrichtung}, die \textbf{teil- oder vollautomatisch} Dienstleistungen verrichtet.
	\item \textbf{Dienstleistungen} sind dabei Tätigkeiten, die nicht der direkten industriellen ERzeugung von Sachgütern, sondern der Verrichtung von \textbf{Leistungen für Menschen und Einrichtungen} dienen.
	\item Einteilung in zwei Klassen
	
\end{itemize}
\subsection{Klassen}
\begin{itemize}
	\item Roboter, die für professionellen Einsatzbereich: \textbf{Rettung}, \textbf{Landwirtschaft}, \textbf{Medizin}
	\item Roboter für den Privaten gebrauch: \textbf{Staubsauger}, \textbf{Rasenmäher}, \textbf{Pfleger}
\end{itemize}